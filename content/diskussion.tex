\section{Diskussion}
\label{sec:Diskussion}

Die aus den Messwerten bestimmte Steigung des Plateaus beträgt

\begin{center}
    $s = (0,641 \pm 0,003)$ $\frac{\%}{100\symup{V}}$.
\end{center}

Sie hat lediglich eine Standardabweichung von $0,4 \%$. Die dazu durchgeführte lineare Ausgleichsrechnung
nähert die Messwerte ebenfalls gut an. Es fällt jedoch auf, dass bei großem $U$ die Messwerte immer weiter von der Gerade divergieren,
allerdings sowohl oberhalb als auch unterhalb, sodass hier von statistischen Fehlern ausgegangen wird.

Die Totzeit wird mit den beiden Methoden zu 

\begin{align*}
    T_\text{Doppel} &= (110 \pm 30) \, \symup{µs}\\
    T_\text{Osz} &= (30 \pm 10) \, \symup{µs}
\end{align*}

bestimmt. Beide Werte haben eine hohe Standardabweichung von $27,3 \%$ bei der Zwei-Quellen-Methode und $33,3 \%$ bei der Oszilloskopmethode.
Des Weiteren weichen beide Werte sehr stark voneinander ab. Der Wert, welcher durch die Zwei-Quellen-Methode berechnet wurde, ist fast das vierfache des anderen.
Die Oszilloskopmethode wird jedoch als ungenauer eingestuft, da das Bild niemals konstant ist, sodass schon die Momentaufnahme eine Näherung ist.
Des Weiteren muss mit analogen Methoden, eine Theoriekurve auf dem Bild einzufügen, um so eine Totzeit bestimmen zu können.
Die Skalierung am Oszilloskop ist ebenfalls sehr ungenau.
Die Methode mit zwei Proben unterliegt lediglich dem statistischen Fehler einer Poisson-Verteilung, welcher aufgrund der einfachen und kurzen Messung hoch ist.
Jedoch kann dadurch angenommen werden, dass der echte Wert der Totzeit in dem Bereich von diesem Wert liegt.

Die bestimmten freigesetzten Ladungsmengen sind

\begin{align*}
    Q_{380 \symup{V}} &= (1,70 \pm 0,02) \cdot 10^{10} e_0\\
    Q_{520 \symup{V}} &= (3,35 \pm 0,03) \cdot 10^{10} e_0\\
    Q_{680 \symup{V}} &= (4,91 \pm 0,05) \cdot 10^{10} e_0.
\end{align*}

Die Standardabweichungen sind allesamt sehr gering, was durch die hohe Dauer der einzelnen Messung erklärt werden kann.
Jedoch muss beachtet werden, dass das Mikroamperemeter lediglich eine analoge Skala besitzt.
Des Weiteren sind die beobachteten Ströme sehr gering, sodass nur bei den drei oben genannten Spannungen unterschiedliche Stromstärken sinnvoll angegeben werden können.
