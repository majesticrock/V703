\section{Zielsetzung}
Ziel des Versuches ist es einerseits die Charakteristik eines Geiger-Müller-Zählrohres zu bestimmen und andererseits die Totzeit desselbigen
mittels unterschiedlicher Methoden zu bestimmen. 
\section{Theorie}
\label{sec:Theorie}
Mit einem Geiger-Müller-Zählrohr lässt sich ionisierte Strahlung messen. Dabei tritt ein ionisiertes $\symup{\alpha}$- oder $\symup{\beta}$-
Teilchen in das Zählrohr ein und löst einen elektrischen Impuls aus. Das Zählrohr besteht dabei aus einem Aufbau, wie er in \autoref{fig:zaehlrohr}
schematisch dargestellt ist. Es ist mit einem Gasgemisch aus Argon und Ethylalkohol. Durch Anlegen einer Spannung zwischen den Anodendraht und
der Zylinderkathode erfahren geladene Teilchen eine Beschleunigung zum Draht hin. Diese kann für beliebig kleine Radien des Drahtes
$r_{\symup{a}}$ zum Draht hin beliebig groß werden.
Wenn nun ein geladenes Teilchen (beispielsweise $\symup{\alpha}$- oder $\symup{\beta}$-Teilchen) in das Zählrohr eintritt, bewegt es sich so 
lange durch das Gas, bis die Energie des Teilchens durch Ionisationsprozesse aufgebraucht ist. Die Anzahl der herausgelösten Elektronen
ist dabei proportional zur Energie des einfallenden Teilchens.
Das Verhalten des Zählrohres nach dieser Primärionisation hängt dabei von der angelegten Spannung ab. Dies ist in \autoref{fig:spannung}
veranschaulicht.